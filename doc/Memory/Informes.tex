\pagestyle{empty}
\topmargin=0cm

César Hervás Martínez, Catedrático de Universidad en Ciencias de la Computación e Inteligencia Artificial del Dpto. de Informática y Análisis Numérico en la Escuela Politécnica Superior de la Universidad de Córdoba.

\vspace{2cm}


\begin{LARGE}
\underline{\textbf{Informa:}}
\end{LARGE}
\\

Que el presente proyecto fin de carrera titulado ``\textit{Modelos de Redes Neuronales para Regresión Ordinal basados en Técnicas de Descenso por Gradiente}", constituye la memoria presentada por D. Raúl Pérula Martínez para aspirar al título de Ingeniero en Informática, ha sido realizado bajo mi dirección en la Escuela Politécnica Superior de la Universidad de Córdoba reuniendo, a mi juicio, las condiciones necesarios exigidas en este tipo de trabajos.\\

Y para que conste, se expide y firme el presente certificado en Córdoba, Septiembre de 2011.

\vspace{2cm}

El Director de proyecto:

\vspace{3.5cm}

Fdo: Prof. Dr. D. César Hervás Martínez


%PAGINA EN BLANCO
\newpage{\pagestyle{empty}\cleardoublepage}

Pedro Antonio Gutiérrez Peña, Profesor Ayudante Doctor del Dpto. de Informática y Análisis Numérico en la Escuela Politécnica Superior de la Universidad de Córdoba.

\vspace{2cm}


\begin{LARGE}
\underline{\textbf{Informa:}}
\end{LARGE}
\\

Que el presente proyecto fin de carrera titulado ``\textit{Modelos de Redes Neuronales para Regresión Ordinal basados en Técnicas de Descenso por Gradiente}", constituye la memoria presentada por D. Raúl Pérula Martínez para aspirar al título de Ingeniero en Informática, ha sido realizado bajo mi dirección en la Escuela Politécnica Superior de la Universidad de Córdoba reuniendo, a mi juicio, las condiciones necesarios exigidas en este tipo de trabajos.\\

Y para que conste, se expide y firme el presente certificado en Córdoba, Septiembre de 2011.

\vspace{2cm}

El Director de proyecto:

\vspace{3.5cm}

Fdo: Prof. Dr. D. Pedro Antonio Gutiérrez Peña
