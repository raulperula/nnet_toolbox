\chapter{Introducción}
	
	Este capítulo tiene como finalidad dar al lector una visión general del dominio del problema. Para ello se le intenta situar en el contexto del mismo, introduciéndole en el modelado de sistemas mediante el entrenamiento de redes neuronales artificiales haciendo uso de algoritmos de regresión ordinal.\\

	Éste es uno de los principales campos de investigación del grupo AYRNA\footnote{http://www.uco.es/grupos/ayrna/} (Aprendizaje y Redes Neuronales Artificiales) del Departamento de Informática y Análisis Numérico de la Universidad de Córdoba (UCO). Grupo en el cual queda situado y que marcará los objetivos del presente proyecto.\\
	
	En el mundo real surge la necesidad de predecir datos que necesitamos o determinados sucesos que ocurren a partir del estudio de determinadas variables, es decir, es necesario anticiparnos a ellos para actuar en consecuencia. Para ello se utiliza el modelado de sistemas, el cual se presenta como uno de los problemas de mayor interés en numerosas ramas de la ciencia. Los modelos son abstracciones de la realidad que el ser humano realiza para llegar a un mayor grado de comprensión de ésta. Una solución al problema de predicción de un suceso es el modelado del mismo, es decir, realizar una abstracción del mismo que recoja sus características y comportamiento, para llegar a la comprensión de su realidad. El modelado de sistemas se puede aplicar en diversos campos de estudio científico. Para ello se realiza un modelado del sistema de manera que el modelo sea capaz de generalizar ante cualquier nueva situación, estudiada o no previamente, siempre dentro del dominio del problema. Para llegar a este modelo se realiza un entrenamiento previo del sistema a partir de los datos observados.\\

	Algunos de los problemas más comunes con los que nos encontramos son la regresión y la clasificación. La regresión trata de establecer una relación funcional entre alguna de las variables dependientes que afectan al problema y que están en escala de ratios o intervalos, y las variables independientes del mismo, mientras que la clasificación intenta predecir la clase de pertenencia de los patrones a los que se aplica, que en este caso las variables dependientes podrán estar en escala nominal u ordinal.\\

	La resolución de estos problemas se ha abordado clásicamente usando técnicas de optimización para minimizar una determinada función de error, previo establecimiento por parte del investigador del tipo de modelo a aplicar. Pero en muchas ocasiones, el modelo a aplicar no es lineal y además suele presentar una alta dimensionalidad en las variables independientes, lo que complica considerablemente el proceso de modelado.\\

	Una de las alternativas más utilizadas en los últimos años para la resolución de este tipo de problemas ha sido la aplicación de las llamadas Redes Neuronales Artificiales.\\

	Hoy día, el ser humano se enfrenta a problemas que debe resolver con éxito ya que, el simple hecho de distinguir entre un perro y un gato en una imagen es una tarea que un niño de preescolar haría fácilmente. Pero, sin embargo, esta misma tarea podría confundir a un ordenador. Para llegar a una solución a estos problemas, el ser humano hace uso de sus capacidades físicas con el fin de resolverlo.\\

	La Inteligencia Artificial trabaja desde hace años en el campo de las Redes Neuronales Artificiales con el fin de simular las capacidades físicas humanas para resolver problemas, ya sean cotidianos o no. Así, están consideradas un paradigma de aprendizaje y procesamiento automático inspirado en la forma en que funciona el sistema nervioso de los animales. Se trata de un sistema de interconexión de neuronas en una red que colabora para producir un estímulo de salida. Dentro de esta vía de investigación se sitúa una parte del grupo de investigación AYRNA de la Universidad de Córdoba, concretamente, en la utilización de Redes Neuronales Evolutivas para la resolución de problemas de modelado.\\

	Con el objetivo de satisfacer las necesidades de este área de investigación, el grupo de investigación AYRNA ha desarrollado de forma teórica varios algoritmos y en concreto el que nos atañe, para su posterior implementación y prueba de rendimiento y de este modo, aportar conocimientos a la comunidad científica.
