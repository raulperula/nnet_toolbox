\chapter{Conclusiones y Futuras Mejoras}
	
	Una vez concluido el desarrollo y la experimentación del proyecto, se realizará una exposición de las conclusiones que se han extraído. Estas conclusiones irán en relación con los objetivos planteados al principio del desarrollo especificados en el Capítulo de Objetivos y con los resultados obtenidos durante la fase de experimentación.
	
	\section{Conclusiones}
		
		En esta sección se va a considerar cuales de los objetivos se han alcanzado en la realización del Proyecto. Una vez finalizado el desarrollo de la implementación del algoritmo de regresión ordinal para redes neuronales y la aplicación gráfica, se puede concretar que se han cumplido los requisitos especificados al comienzo del Proyecto.\\
		
		Los objetivos alcanzados han sido los siguientes:
		
		\begin{itemize}
			\item Se ha desarrollado la implementación del algoritmo propuesto de forma teórica para regresión ordinal basado en redes neuronales artificiales.
			\item Se ha realizado una comparativa de los resultados obtenidos basado en efectividad de la clasificación obtenida por el algoritmo nominal y el nuevo algoritmo basado en la ordinalidad de las clases.
			\item Se ha desarrollado una herramienta software para el tratado y la obtención de resultados a partir de un conjunto de datos, aplicando el algoritmo implementado basado en redes neuronales y haciendo uso del toolbox de Matlab, \textit{nnet}. Dicha herramienta ha sido implementada optimizando al máximo tanto la precisión de los modelos generados como el tiempo de computación necesario para obtener los resultados y que es algo muy importante en algoritmos de este estilo.
			\item Se ha modularizado todas los módulos y funciones para que tenga una compatibilidad completa con el toolbox nnet de Matlab.
			\item Se ha realizado un diseño experimental para cada uno de los problemas considerados, ajustando los parámetros de los experimentos y extrayendo las conclusiones pertinentes. Los modelos obtenidos como solución a dichos problemas han obtenido unos resultados muy buenos en clasificación, mejorando en un porcentaje variable para cada caso específico los resultados que hasta ahora se habían obtenido con modelos de clasificación nominal.
		\end{itemize}
		
		Por todos estos motivos, considerando que el proyecto que se ha desarrollado ha conseguido abarcar todas las metas que se propusieron al inicio del mismo, es un orgullo y una satisfacción haber aportado al grupo de investigación AYRNA, en sus estudios sobre aprendizaje e inteligencia artificial con ayuda de las redes neuronales artificiales, la implementación y comprobación del algoritmo desarrollado de forma teórica y la aportación de una interfaz gráfica completa y compatible con la librería de Matlab.
		
	\section{Futuras Mejoras}
		
		En esta sección se expondrán varias ideas con posibles mejoras a alcanzar en futuras ampliaciones del trabajo desarrollado.\\
		
		Como se ha expuesto a lo largo del desarrollo del presente proyecto, el objetivo principal era el desarrollo de la implementación del algoritmo de regresión ordinal y la comprobación de la efectividad del mismo en comparación con el método nominal.\\
		
		Dicho esto, se podría ver como una posible mejora del proyecto el estudio continuo de nuevos algoritmos basados en regresión ordinal y a partir de estos la nueva implementación de otro algoritmo que mejorase la efectividad de los resultados.\\
		
		Otra posible mejora sería el aumento de la funcionalidad de la herramienta software, por ejemplo, añadiendo la funcionalidad para la realización del método nominal o algún otro método basado en redes neuronales.\\
		
		Por último, otra posible mejora que podría realizarse es la adaptación, en alguno de los sentidos, al toolbox nnet de Matlab del resto de algoritmos o desarrollos que el grupo de investigación AYRNA tiene de proyectos o estudios anteriores basados en redes neuronales artificiales, completando así dicho paquete de software computacionalmente muy potente.
