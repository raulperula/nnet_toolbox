\chapter{Objetivos}
	
	En el presente capítulo se expondrán todos los objetivos funcionales que se pretenden alcanzar con el desarrollo de este proyecto.\\
	
	El propósito de este proyecto es desarrollar un algoritmo de regresión ordinal basado en redes neuronales, mediante la utilización de una función de \textit{ranking} $f(x)$ formada por la suma ponderada de un conjunto de funciones de base (incluyendo funciones de tipo sigmoide, unidad producto, funciones de base radial, etc.).\\

	En concreto, los objetivos perseguidos son los siguientes:

	\begin{enumerate}
		\item Realizar un estudio teórico sobre las temáticas relacionadas con el proyecto, incluyendo la regresión ordinal, la clasificación tradicional y los algoritmos de descenso por gradiente.
		\item Implementar el algoritmo iRPROP+ (el cuál es una mejora del algoritmo RPROP), utilizando como base la implementación incluida en el paquete de redes neuronales (\verb'nnet') del software Matlab \cite{Matlab}.
		\item Desarrollar un algoritmo de regresión ordinal basado en Redes Neuronales. Este objetivo engloba varios sub-objetivos:
		\begin{enumerate}
	  		\item Modificación del modelo funcional estándar de redes neuronales para clasificación. Esto supone considerar una red neuronal de una sola neurona en capa de salida y aplicar una transformación de la salida que aproxime un valor de probabilidad de pertenencia a cada una de las clases. Esta transformación la basaremos en la Regresión Logística Ordinal estándar (\textit{Proportional Odd Model}, \cite{Mcc80}).
			\item Modificación de la función de error utilizada como base en el algoritmo iRProp+.
			\item Evaluar el algoritmo en un conjunto de bases de datos de prueba, comparar los resultados con los obtenidos por otros clasificadores ordinales y reformular la hipótesis y/o el algoritmo inicial, intentado conseguir (siempre que fuese posible) unos resultados competitivos.
		\end{enumerate}
		\item Implementar una interfaz gráfica que facilite la utilización del algoritmo implementado.
\end{enumerate}
