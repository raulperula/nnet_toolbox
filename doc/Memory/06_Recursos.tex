\chapter{Recursos}
	
	En este capítulo se expondrán de forma clara y concisa los recursos humanos y materiales necesarios para este proyecto. Los recursos se definen como aquellos medios de los que se dispone para abordar el proceso de desarrollo del proyecto. El análisis de los recursos existentes se realizará atendiendo a una doble perspectiva:
	
	\begin{itemize}
		\item \textbf{Recursos humanos:} son aquellos que están constituidos por toda persona que intervenga en el proceso de desarrollo del sistema.
		\item \textbf{Recursos materiales:} son aquellos que pueden definirse como el conjunto de todas las entidades no animadas que permiten realizar el proceso de desarrollo de la aplicación, así como la generación de la documentación relativa a la misma.
	\end{itemize}
	
	\section{Recursos humanos}
	
		El conjunto de personas que intervendrán durante el proceso de desarrollo del presente proyecto se muestran a continuación:
	
		\begin{itemize}
			\item \textbf{Directores:}
			\begin{itemize}
				\item Prof. Dr. Pedro Antonio Gutiérrez Peña.\\
	
				Profesor Ayudante Doctor del Dpto. de Informática y Análisis Numérico y miembro investigador del grupo AYRNA.
	
				\item Prof. Dr. César Hervás Martínez.\\

				Catedrático del Dpto. de Informática y Análisis Numérico y director del grupo AYRNA.
			\end{itemize}
			
			Los directores se encargarán de supervisar las tareas de desarrollo para comprobar que los resultados obtenidos se corresponden con los requisitos planteados. Además, facilitarán aquellos recursos materiales que resulten necesarios para abordar con éxito el proceso de desarrollo.
		
			\item \textbf{Autor:}
			\begin{itemize}
				\item Raúl Pérula Martínez.\\
	
				Ingeniero Técnico Informático de Sistemas.
			\end{itemize}
		\end{itemize}
	
	\section{Recursos materiales}
		
		\subsection{Recursos software}
			
			\begin{itemize}
				\item Sistema operativo Ubuntu 10.04.
				\item Lenguaje de programación y herramienta MATLAB 2010a.
				\item Lenguaje \LaTeX{} para la realización de la documentación.
				\item Texmaker 2.1 como editor de documentos \LaTeX{}.
				\item Programa de edición de diagramas, dia 0.97.1.
			\end{itemize}
			
		\subsection{Recursos hardware}
			
			Equipo portátil con las siguiente características técnicas:
			
			\begin{itemize}
				\item Procesador Intel Core 2 Duo T7300 (Santa Rosa) a 2 GHz (4 MB memoria caché L2).
				\item 2048 MB de memoria RAM.
				\item Disco duro de 120 GB.
			\end{itemize}
