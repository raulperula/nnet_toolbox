%PAQUETES

\usepackage[left=3.5cm,top=2.5cm,right=2cm,bottom=2.5cm]{geometry}
\usepackage[pdftex]{graphicx}	% paquete para imagenes
\usepackage{latexsym,amsmath,amssymb,amsfonts,cancel}
\usepackage[activeacute,spanish]{babel} 	% paquete para tildes
\usepackage[utf8]{inputenc}		% paquete para español
\usepackage{multirow}
%\usepackage{multicol}	% paquete para poder poner texto multicolumna
%\usepackage{pdflscape}	% paquete para poder poner paginas horizontales
\usepackage{rotating}	% paquete para rotaciones

\usepackage{hyperref}	% paquete para hiperenlaces
\hypersetup{
	bookmarks=false,
	pdfborder={0 0 0},
	colorlinks=true,
	linkcolor=black,
	citecolor=black,
	urlcolor=blue
}

\usepackage{csvtools}	% paquete para cargar datos desde fichero csv
\usepackage{csvsort}	% paquete para ordenar tablas csv

\usepackage{color, colortbl}
\definecolor{light-gray}{gray}{0.95}
\definecolor{myred}{RGB}{252,74,58}
\definecolor{mygreen}{RGB}{111,255,79}

\usepackage{listingsutf8}
% CARACTERISTICAS DE LOS ARCHIVOS DE CODIGO FUENTE
\lstset{
	language=Matlab,
	basicstyle=\small,
	keywordstyle=\color[rgb]{0.5,0,0}\bfseries,
	stringstyle=\color{blue},
	commentstyle=\color[rgb]{0,0.5,0},
	showstringspaces=false,
	numbers=left,
	numberstyle=\tiny,
	%stepnumber=5,
	numbersep=5pt,
	backgroundcolor=\color{light-gray},
	showspaces=false,
	showtabs=false,
	frame=single,
	framexleftmargin=10mm,
	frame=trBL,
	rulesepcolor=\color{black},
	tabsize=2,
	captionpos=b,
	breaklines=true,
	breakatwhitespace=false,
	inputencoding=utf8,
	extendedchars=\true
}

\DeclareGraphicsExtensions{.bpm,.png,.pdf,.jpg}	% extensiones de imgenes

% RENOMBRAMIENTO DE COMANDOS

\newcommand{\sen}{\mathtop{\rm sen}\nolimits}
\newcommand{\arcsen}{\mathtop{\rm arcsen}\nolimits}
\newcommand{\arcsec}{\mathtop{\rm arcsec}\nolimits}

\def\max{\mathtop{\mbox{\rm mx}}}
\def\min{\mathtop{\mbox{\rm mn}}}

%\def\changemargin#1#2{\list{}{\rightmargin#2\leftmargin#1}\item[]}
%\let\endchangemargin=\endlist

%Para personalizar el formato de los titulos
\usepackage{pstcol}
\makeatletter
\def\LigneVerticale{\vrule height 4cm depth 2cm\hspace{0.1cm}\relax}
\def\LignesVerticales{%
  \let\LV\LigneVerticale\LV\LV\LV\LV\LV\LV\LV\LV\LV\LV}
\def\GrosCarreAvecUnChiffre#1{%
  \rlap{\vrule height 0.8cm width 1cm depth 0.2cm}%
  \rlap{\hbox to 1cm{\hss\mbox{\white #1}\hss}}%
  \vrule height 0pt width 1cm depth 0pt}
\def\@makechapterhead#1{\hbox{%
    \huge 
    \LignesVerticales
    \hspace{-0.5cm}%
    \GrosCarreAvecUnChiffre{\thechapter}
    \hspace{0.2cm}\hbox{#1}%
}\par\vskip 2cm}
\def\@makeschapterhead#1{\hbox{%
    \huge 
    \LignesVerticales
    \hspace{0.5cm}
    \hbox{#1}%
}\par\vskip 2cm}

%Para conseguir que en las páginas en blanco no ponga cabeceras
\makeatletter
\def\clearpage{%
  \ifvmode
    \ifnum \@dbltopnum =\m@ne
      \ifdim \pagetotal <\topskip
        \hbox{}
      \fi
    \fi
  \fi
  \newpage
  \thispagestyle{empty}
  \write\m@ne{}
  \vbox{}
  \penalty -\@Mi
}
\makeatother