\chapter{Restricciones}
	
	En este capítulo se expondrán todas las restricciones, o factores limitativos, existentes en el ámbito del diseño y que condicionan la elección de una u otra alternativa. Los factores limitativos pueden estructurarse en dos grupos:
	
	\begin{itemize}
		\item \textbf{Factores dato:} son aquellos que no pueden ser modificados durante el transcurso del proyecto, como puede ser el presupuesto económico asignado al proyecto o la duración estimada del mismo.
		\item \textbf{Factores estratégicos:} representan variables de diseño que permiten la elección entre diferentes alternativas por parte del ingeniero. En función de la opción escogida, podrá alterarse el proceso de desarrollo y el propio producto final obtenido, con lo que resultará necesario analizar las posibilidades existentes en las primeras etapas del proceso.
	\end{itemize}
	
	\section{Factores dato}
	
		En el desarrollo de este proyecto se van a considerar los siguientes factores dato impuestos por el tipo de proyecto:
	
		\begin{itemize}
			\item \textbf{Restricciones humanas:} al ser éste un proyecto final de carrera, este factor lo condiciona el director de proyecto (perteneciente al grupo de investigación AYRNA), restringiendo el proyecto a un número determinado de personas. En nuestro caso, un solo alumno para el desarrollo del mismo.
			\item \textbf{Restricciones temporales:} estará condicionado al cumplimiento de los objetivos previamente establecidos, aunque por lo general, se suele establecer limitaciones con el fin de no demorar en exceso la presentación del Proyecto. En este caso, el tiempo de elaboración de este Proyecto se espera que no sobrepase la convocatoria de septiembre del curso 2010/2011, a expensas de no sufrir contratiempos.
			\item \textbf{Restricciones hardware:} dado a la cantidad de cálculos que realizarán los algoritmos desarrollados, desde el cliente nos imponen ciertas restricciones hardware (como la potencia de los equipos en los que se utilizará el sistema) a tener en cuenta a la hora de elegir ciertos parámetros en el futuro de diseño de la aplicación.
			\item \textbf{Restricciones software:} según las características del proyecto y el fin que tendrá, meramente investigador. No se impone un software específico ni un lenguaje de programación en especial, pero sí que tenga la suficiente potencia para desarrollar el proyecto.
		\end{itemize}
		
	\section{Factores estratégicos}
	
		Los principales factores estratégicos que afectan al presente proyecto, así como los diferentes motivos que justifican la elección realizada en cada restricción, son los que se comentan a continuación:
		
		\begin{itemize}
			\item Los algoritmos que se desea desarrollar deberán estar bien modularizados, de manera que permita de manera fácil y con la mayor agilidad posible la realización de futuras modificaciones y ampliaciones que puedan considerarse necesarias.
			\item Los algoritmos se desarrollarán en el entorno de desarrollo Matlab, porque la potencia que ofrece este software de desarrollo matemático y la utilidad de tener a disposición un Toolbox sobre redes neuronales que facilitan la elaboración del proyecto.
			\item Para la elaboración de la documentación en \LaTeX{} se utilizará el editor Texmaker, ya que es de libre distribución.
			\item Para la elaboración de cualquier tipo de diagramas se hará uso de la herramienta Día, disponible en libre distribución para los distintos sistemas operativos.
		\end{itemize}
